% !TeX encoding   = UTF-8
\documentclass[12pt]{article}

\usepackage{sbc-template}

\usepackage{graphicx,url}
\usepackage[brazil]{babel}
\usepackage[utf8]{inputenc}
\usepackage{graphicx}   %Package para figuras
\usepackage{enumerate}
\usepackage{tabularx}
\usepackage{multirow}
\usepackage[table,xcdraw]{xcolor}


\sloppy

\title{Refatorações Arquiteturais: Um estudo sobre o efeito de Mover Classe na Arquitetura de Sistemas}

\author{Vagner Clementino\inst{1}}

\address{Departamento de Ciência da Computação\\
        Universidade Federal de Minas Gerais (UFMG)\\
  \email{vagnercs@dcc.ufmg.br}
}

\date{Maio de 2016}
\begin{document}

\maketitle

%\begin{abstract}
%  This meta-paper describes the style to be used in articles and short papers
%  for SBC conferences. For papers in English, you should add just an abstract
%  while for the papers in Portuguese, we also ask for an abstract in
%  Portuguese (``resumo''). In both cases, abstracts should not have more than
%  10 lines and must be in the first page of the paper.
%\end{abstract}

\begin{resumo}
 \textbf{TODO}
\end{resumo}


\section{Introdução}
\label{sec:intro}

A atividade de refatoração têm por objetivo alterar o código fonte de um software sem modificar o seu comportamento. Em última instância o objetivo de refatorar é melhorar a qualidade interna do sistema\cite{1999:RID:311424,Opdyke:1992:ROF:169783}. Sua importância é reconhecida tanto na literatura quanto na indústria, no qual, nesta última, é possível verificar a proposição de processos de desenvolvimento de software ágeis que incorporam a refatoração como atividade rotineira \cite{Beck:2000:PEP:557458}. No últimos anos, pesquisas foram realizadas com o objetivo de entender com qual frequência os desenvolvedores aplicam diferentes tipos de refatoração \cite{Murphy-Hill:2009:WRW:1555001.1555044}, a sua relação entre a correção de \textit{bugs}\cite{Kim:2011:EIR:1985793.1985815} e testes \cite{Kim:2012:EII:2473496.2473590} e a percepção da mesma pelos desenvolvedores\cite{Kim:2012:FSR:2393596.2393655}.

No últimos anos estudos vêm focando em entender as motivações por detrás da refatoração. Existe um consenso que a motivação original é remover códigos de baixa qualidade conhecidos como \textit{Bad Smells}\cite{1999:RID:311424}. Todavia, estudos demonstraram que o desenvolvedores utilizam a refatoração para outros fins. Por exemplo, foi encontrado indícios que a refatoração \textit{Extrair Método} foi utilizada para fins de extensão do sistema \cite{Tsantalis2013}, para reutilização do código \cite{Danilo}, dentre outras motivações.

Apesar da existência de estudos relativos à motivação da refatoração, ao bem do nosso conhecimento, não existem trabalhos que relacionem a atividade de refatorar com mudanças na arquitetura do sistema. Ou seja, \textit{refatorações que têm por objetivo alterar a arquitetura do software, reorganizar o código existente em uma nova camada lógica e cujo objetivo principal é aumentar a qualidade geral dos software}.



\bibliographystyle{sbc}
\bibliography{../bib/paper-arqsw}

\end{document}
