% !TeX encoding   = UTF-8
\documentclass[12pt]{article}

\usepackage{sbc-template}

\usepackage{graphicx,url}
\usepackage[brazil]{babel}
\usepackage[utf8]{inputenc}
\usepackage{graphicx}   %Package para figuras
\usepackage{enumerate}
\usepackage{tabularx}
\usepackage{multirow}
\usepackage[table,xcdraw]{xcolor}


\sloppy

\title{Refatorações Arquiteturais: Um estudo sobre o efeito de Mover Classe na Arquitetura de Sistemas}

\author{Vagner Clementino\inst{1}}

\address{Departamento de Ciência da Computação\\
        Universidade Federal de Minas Gerais (UFMG)\\
  \email{vagnercs@dcc.ufmg.br}
}

\date{Maio de 2016}
\begin{document}

\maketitle

%\begin{abstract}
%  This meta-paper describes the style to be used in articles and short papers
%  for SBC conferences. For papers in English, you should add just an abstract
%  while for the papers in Portuguese, we also ask for an abstract in
%  Portuguese (``resumo''). In both cases, abstracts should not have more than
%  10 lines and must be in the first page of the paper.
%\end{abstract}

\begin{resumo}
 \textbf{TODO}
\end{resumo}


\section{Introdução}
\label{sec:intro}

A atividade de refatoração têm por objetivo alterar o código fonte de um software sem modificar o seu comportamento. Em última instância o objetivo de refatorar é melhorar a qualidade interna do sistema\cite{1999:RID:311424,Opdyke:1992:ROF:169783}. Sua importância é reconhecida tanto na literatura quanto na indústria, no qual, nesta última, é possível verificar a proposição de processos de desenvolvimento de software ágeis que incorporam a refatoração como atividade rotineira \cite{Beck:2000:PEP:557458}. No últimos anos, pesquisas foram realizadas com o objetivo de entender com qual frequência os desenvolvedores aplicam diferentes tipos de refatoração \cite{Murphy-Hill:2009:WRW:1555001.1555044}, a sua relação entre a correção de \textit{bugs}\cite{Kim:2011:EIR:1985793.1985815} e testes \cite{Kim:2012:EII:2473496.2473590} e a percepção da mesma pelos desenvolvedores\cite{Kim:2012:FSR:2393596.2393655}.

No últimos anos estudos vêm focando em entender as motivações que levam os desenvolvedores a realizem refatoração. Existe um consenso que a motivação original é remover porções de código com baixa qualidade conhecidos como \textit{Bad Smells}\cite{1999:RID:311424}. Todavia, estudos demonstraram que o desenvolvedores refatoram para outros fins, como por exemplo, a refatoração \textit{Extrair Método} que pode ser utilizada para fins de extensão do sistema \cite{Tsantalis2013}ou ainda para reutilização do código \cite{Danilo}, dentre outras motivações.

Apesar da existência de estudos relativos à motivação da refatoração, ao bem do nosso conhecimento, não existem trabalhos que relacionem a atividade de refatorar com mudanças na arquitetura do sistema. Ou seja, refatorações que têm por objetivo alterar a arquitetura do software, reorganizar o código existente em uma nova camada lógica ou cujo objetivo é tratar o problema da  \textit{Erosão Arquitetural}. O processo de Erosão arquitetural e conhecido como os desvios ocorridos no código de um sistema que causam violação de alguma regra arquitetural previamente estabelecida\cite{Perry:1992:FSS:141874.141884}. Violações na arquitetura podem ser de dois tipos: \textit{divergência}, quando uma dependência que existe no código fonte viola a arquitetura planejada; \textit{ausência}, caso onde o código fonte não estabelece uma dependência que é prescrita na arquitetura planejada \cite{5204070}.	

Neste sentido, este trabalho se propõe em analisar a relação entre mudanças na arquitetura de um sistema e a refatoração \textit{Mover Classe}. A fim de investigar tal relação iremos analisar diversas versões de 03 sistemas de código aberto desenvolvidos em Java visando responder as seguintes questões de pesquisa:


\begin{description}
	\item[RQ1] Com qual frequência a refatoração Mover Classe tem por objetivo alterar a arquitetura do sistema?
	\item[RQ2] Com qual frequência o desenvolvedor informa que a refatoração teve por objetivo alterar a arquitetura do sistema?
	\item[RQ3] A refatoração Mover Classe prevalece na resolução de desvios arquiteturais do tipo divergência ou ausência?	
\end{description}

Ao responder as questões proposta neste trabalho entendemos que iremos contribuir no aumento do entendimento das razões que levam os desenvolvedores a realizar refatorações. Esta informação poderá ser utilizada posteriormente na construção de ferramentas que ajudem os times de desenvolvimento em tarefe relativas à mudança da arquitetura do software. Além disso será possível verificar a relação entre a atividade de refatoração e a mudança de arquitetura de um sistema. 


O restante deste trabalho está organizado da seguinte forma: a Seção \ref{sec:metodologia} descreve a metodologia utilizada neste estudo; a Seção \ref{sec:resultados} apresenta os resultados e responde as questões de pesquisas propostas; na Seção \ref{sec:ameacas} realiza-se uma discussão sobre as ameaças à validade do trabalho; a Seção \ref{sec:trabalhos-relacionados} apresenta os trabalhos relacionados à análise da atividade e detecção de refatoração; a Seção \ref{sec:conclusao} sumariza o artigo e discute suas principais contribuições.

\section{Metodologia}
\label{sec:metodologia}

\section{Resultados}
\label{sec:resultados}

\section{Ameaças à Validade}
\label{sec:ameacas}

\section{Trabalhos Relacionados}
\label{sec:trabalhos-relacionados}

\section{Conclusão}
\label{sec:conclusao}
\bibliographystyle{sbc}
\bibliography{../bib/paper-arqsw}

\end{document}
