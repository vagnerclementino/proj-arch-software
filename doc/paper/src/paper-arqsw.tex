% !TeX encoding   = UTF-8
\documentclass[12pt]{article}

\usepackage{sbc-template}

\usepackage{graphicx,url}
\usepackage[brazil]{babel}
\usepackage[utf8]{inputenc}
\usepackage{graphicx}   %Package para figuras
\usepackage{enumerate}
\usepackage{tabularx}
\usepackage{multirow}
\usepackage[table,xcdraw]{xcolor}


\sloppy

\title{Refatorações Arquiteturais: Um estudo sobre Mover Classe na Arquitetura de Sistemas}

\author{Vagner Clementino\inst{1}}

\address{Departamento de Ciência da Computação\\
        Universidade Federal de Minas Gerais (UFMG)\\
  \email{vagnercs@dcc.ufmg.br}
}

\date{Maio de 2016}
\begin{document}

\maketitle

%\begin{abstract}
%  This meta-paper describes the style to be used in articles and short papers
%  for SBC conferences. For papers in English, you should add just an abstract
%  while for the papers in Portuguese, we also ask for an abstract in
%  Portuguese (``resumo''). In both cases, abstracts should not have more than
%  10 lines and must be in the first page of the paper.
%\end{abstract}

\begin{resumo}
 \textbf{TODO}
\end{resumo}


\section{Introdução}
\label{sec:intro}

A atividade de refatoração têm por objetivo altera o código fonte de um software sem com isso alterar o seu comportamento visando em última instância melhorar a qualidade interna do sistema\cite{1999:RID:311424,Opdyke:1992:ROF:169783}. Sua importância é reconhecida tanto na literatura quanto na indústria, no qual processos de desenvolvimento ágeis o incorporam como atividade rotineira \cite{Beck:2000:PEP:557458}. No últimos anos, pesquisas foram realizadas com o objetivo de entender com qual frequência os desenvolvedores aplicam diferentes tipos de refatoração \cite{Murphy-Hill:2009:WRW:1555001.1555044}, relação com a correção de \textit{bugs}\cite{Kim:2011:EIR:1985793.1985815} e testes \cite{Kim:2012:EII:2473496.2473590}, e sua percepção pelos desenvolvedores\cite{Kim:2012:FSR:2393596.2393655}.


No últimos anos alguns estudos vêm focando em entender as motivações por detrás da refatoração. A motivação original é a de remover padrões de código de baixa qualidade conhecidos como \textit{Bad Smells}\cite{1999:RID:311424}. Todavia, estudos demonstraram que o desenvolvedores utilizam a refatoração \textit{Extrair Método} para fins de extensão do sistema \cite{Tsantalis2013} ou ainda para reutilização do código \cite{Danilo}. 

Apesar da existência de estudos relativos à motivação da refatoração, ao bem do nosso conhecimento, não existentes trabalhos que relacionem a atividade de refatorar com mudanças na arquitetura do sistema. Ou seja, Architecture Refactoring aims changing the software architecture, reorganizing the existing code into new logical layers with the main scope of increase the overall software quality and surpass architecture/infrastructure limitations that critically mine systematic qualities as (for example) scalability, extensibility, evolvability, testability and robustness of the actual solution system. A major challenge of this kind of refactoring is trying to reuse as much as possible existing code, to avoid writing a new software. Use of automatic tools can be few help (maybe just to detect circular dependencies), here the most work will come from your brain and experience.

\bibliographystyle{sbc}
\bibliography{../bib/paper-arqsw}

\end{document}
